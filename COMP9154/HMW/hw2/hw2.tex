\documentclass[a4paper,12pt,oneside]{article}
\usepackage{fancyhdr}
\pagestyle{fancy}
\fancyhead[L]{COMP9154}
\fancyhead[C]{Homework 2}
\fancyhead[R]{Yiyan Yang z5183946}
    \title{Homework Week 2}
    \author{Yiyan Yang z5183946}
\begin{document}
    \setlength{\headheight}{14.49998pt}
    \addtolength{\topmargin}{-2.49998pt}
    \maketitle
    \thispagestyle{fancy}
    \section{No skip}
    Because the definition of the LTL logic symbol $\diamond$ is that at some point, 
    the followed statement will be true. 
    At the time when x is equal to 3, 
    $(x == 1\ \mathcal {U}\ x == 3)$ is guaranteed to be true.
    Hence we can conculde that $\diamond (x == 1\ \mathcal {U}\ x == 3)$ 
    is not violated for the given code.
    
\end{document}