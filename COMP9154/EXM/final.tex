\documentclass[a4paper,answers]{exam}
\usepackage{hyperref}
%\usepackage{eulervm}
\usepackage{mathpartir}
%\usepackage{pgffor}
%\usepackage{ascii}
\usepackage{xcolor}
\usepackage{placeins} %for FloatBarrier
\usepackage{version,a4wide}
%,pifont}
%\usepackage{textcomp}
\usepackage{amsmath,amsthm}
\usepackage{stmaryrd}
\usepackage{amssymb}
\usepackage{lipsum}
\renewcommand{\ttdefault}{cmtt}
\pointname{ marks}
\qformat{\textbf{Question \thequestion} $\ $ (\thepoints)\hfill}
\newcommand\llangle{\langle\!\langle}

\usepackage{tikz}
\usetikzlibrary{automata,positioning,shapes.geometric,fit,calc}

\tikzset{initial text={},on grid, auto, thin,
    every state/.style={circle,minimum size=.6cm,
      %draw=blue!50,
      draw=black,
      thin,fill=none,text=black},
    node distance=1.2cm,on grid,auto,
    bend angle=35}
\newcommand\rrangle{\rangle\!\rangle}
\newcommand{\mapstos}[0]{\stackrel{\star}{\mapsto}}
\newcommand{\mapstoc}[0]{\stackrel{!}{\mapsto}}

\begin{document}
\begin{flushright}
   $\ $\\[5em] %Name: $\underline{\qquad\qquad\qquad\qquad\qquad\qquad}$\\[1em]
    %Student Number: $\underline{\qquad\qquad\qquad\qquad\qquad\qquad}$\\[1em]
    %Signature: $\underline{\qquad\qquad\qquad\qquad\qquad\qquad}$\\[3em]
\end{flushright}
\begin{center}
\textsc{\Large The University of New South Wales}\\[3em]
{\Huge COMP3151/9154\\[0.3em] Foundations of Concurrency}\\[3em]
{\Large \textbf{Final Exam}}\\[1em]
{\large \textit{Term 2, 2022}}\\[10em]
{\large
\begin{description}
    \item[Time Allowed:] 24 Hours. Submit by 8AM Sydney time on August 23.
    \item[Total Marks Available:] 100
    \item[] $\!\!\!\!\!$ Brief answers are \textbf{strongly} preferred. Extreme verbosity may cost marks.
    \item[] $\!\!\!\!\!$ Produce a typeset PDF file, via \LaTeX{} or otherwise, with your answers.
    \item[] $\!\!\!\!\!$ Submit with \texttt{give cs3151 exam exam.pdf} or with the \texttt{give} web interface.
    \item[] $\!\!\!\!\!$ The exam is open-book. You may read anything you like, and in general use any passive resource.
    \item[] $\!\!\!\!\!$ You \textbf{may not} use active resources---don't solicit, offer, or accept help of any kind, with one exception: you may ask private questions on Ed. Johannes will monitor Ed regularly, except when he sleeps (around 10PM--6AM).
    \item[] $\!\!\!\!\!$ Include the following statement in your PDF file:\\
\emph{I declare that all of the work submitted for this exam is my own work, completed without assistance from anyone else.}
   \item[] $\!\!\!\!\!$ You must adhere to the UNSW student conduct requirements listed at\\ \url{https://student.unsw.edu.au/conduct}.
\end{description}
}
\end{center}
\newpage

%\setcounter{latest@ques}{1}

\begin{questions}

\section*{Part I}

\question[x]
\lipsum[1-1]

\begin{solution}
  You can put your solutions in \verb!solution! environments like this.
\end{solution}

\question[x]
\lipsum[1-1]

\begin{solution}
  You can put your solutions in \verb!solution! environments like this.
\end{solution}

\question[x]
\lipsum[1-1]

\begin{solution}
  You can put your solutions in \verb!solution! environments like this.
\end{solution}

\question[x]
\lipsum[1-1]

\begin{solution}
  You can put your solutions in \verb!solution! environments like this.
\end{solution}

\FloatBarrier
\section*{Part II}

These questions are about the paper \emph{Simple, Fast, and Practical Non-Blocking and Blocking Concurrent Queue Algorithms} by Maged M. Michael and Michael L. Scott (PODC 1996: 267-275).

\question[x]
\lipsum[1-1]

\begin{solution}
  You can put your solutions in \verb!solution! environments like this.
\end{solution}

\question[x]
\lipsum[1-1]

\begin{solution}
  You can put your solutions in \verb!solution! environments like this.
\end{solution}

\question[x]
\lipsum[1-1]

\begin{solution}
  You can put your solutions in \verb!solution! environments like this.
\end{solution}

\end{questions}

\textbf{Dont forget to submit your work and include the statement given on the front page}.\\

\centerline{\bf \large --- END OF EXAM ---}

\end{document}
